% Copyright 2018  李文威 (Wen-Wei Li).
% Permission is granted to copy, distribute and/or modify this
% document under the terms of the Creative Commons
% Attribution 4.0 International (CC BY 4.0)
% http://creativecommons.org/licenses/by/4.0/

% 目的: 字体相关设置, 呼叫相关宏包.
% 将由 AJbook.cls 引入
% 必须提供 \kaishu, \songti, \heiti, \thmheiti, \fangsong 几种字型切换命令, 在文档类中使用.
\ProvidesFile{font-setup-open.tex}[2018/03/04]

% 设置 xeCJK 字体及中文数字
%\setmainfont{TeX Gyre Pagella}	% 设置西文衬线字体
\setsansfont{Helvetica Neue}
	% 设置西文无衬线字体

% 自用模式: Fandol 字体 + 思源黑体 (Noto Sans CJK SC), 宜留意字体高低差异.
% ---- 中文字体统一用 macOS 自带的苹方 ----
\setCJKmainfont{PingFang SC}
\setCJKsansfont{PingFang SC}
\setCJKmonofont{PingFang SC}

% 各个族名只是“标签”,都绑到 PingFang SC 上
\setCJKfamilyfont{kai}{PingFang SC}        % \kaishu
\setCJKfamilyfont{song}{PingFang SC}       % \songti
\setCJKfamilyfont{fangsong}{PingFang SC}   % \fangsong
\setCJKfamilyfont{hei}{PingFang SC}        % \heiti
\setCJKfamilyfont{hei2}{PingFang SC}       % thmheiti / chapter 标题
\setCJKfamilyfont{sectionfont}{PingFang SC}% section 标题
\setCJKfamilyfont{pffont}{PingFang SC}     % 证明用字体
\setCJKfamilyfont{emfont}{PingFang SC}     % \em 强调用的字体



\defaultfontfeatures{Ligatures=TeX} 
\XeTeXlinebreaklocale "zh"
\XeTeXlinebreakskip = 0pt plus 1pt minus 0.1pt

% 以下设置字体相关命令, 用于定理等环境中.
\newcommand\kaishu{\CJKfamily{kai}} % 楷体
\newcommand\songti{\CJKfamily{song}} % 宋体
\newcommand\heiti{\CJKfamily{hei}}	% 黑体
\newcommand\thmheiti{\CJKfamily{hei2}}	% 用于定理名称的黑体
\newcommand\fangsong{\CJKfamily{fangsong}} % 仿宋
\renewcommand{\em}{\bfseries\CJKfamily{emfont}} % 强调