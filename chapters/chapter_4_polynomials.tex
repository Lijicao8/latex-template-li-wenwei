\chapter{Properties of polynomials}
\section{Divided difference}
对于一个函数 $f$ 和点列
$X : x_{0}, \ldots, x_{n}$ (不要求按 $\leq$ 排序),我们定义 $f$ 的 $n$ 阶差商为
$$[x_{0}, \ldots, x_{n}]f := A_{n}$$
其中 $A_{n}$ 是插值多项式 $P_{n}(f, X; x)$ 的最高次项 $x^{n}$ 的系数。

\paragraph{n-阶差商满足的性质:}
\begin{enumerate}
    \item 若$f$高阶可微,则$[x_0,\dots,x_n]f$是$f^{(k)}(x_i),(0\leq k\leq m_i-1)$的线性组合,其中$m_i$是$x_i$的重数;
    \item $[x_0,\dots,x_n]f$是一个常数,若$f$是一个多项式,更进一步,若多项式的次数$<n$,则$n-$阶差商是0;
    \item$[x_0,\dots,x_0]f = f^{(n)}(x_0)/n!$.
    \item 牛顿型插值公式:\begin{equation*}
        P_n(f,X;x)= \sum_{k=0}^n(x-x_0)\dots(x-x_{k-1})[x_0,\dots,x_k]f;
    \end{equation*} 
    \item 若$f\in C^n[a,b]$,根据Rolle定理\begin{equation*}
        [x_0,\dots,x_n]f = f^{(n)}(\xi)/n!~~\text{for some } a\leq \xi\leq b;
    \end{equation*}
    \item 若$x_i$互异,则
    \begin{align}\label{eq:差分运算是线性组合}
        [x_0,\dots,x_n]f &= \sum_{k=0}^n\frac{f(x_k)}{(x_k-x_0)\cdots(x_k-x_{k-1})(x_k-x_{k+1})\cdots (x_k-x_n)}\notag\\
        & = \sum_{k=0}^n\frac{f(x_k)}{\Omega'(x_k)}
    \end{align}
    \item \begin{equation*}
        \Delta_h^n f(x) = n!h^n[x,x+h,\dots,x+nh]f;
    \end{equation*}
    \item \begin{equation*}
        [x_0,\dots,x_n](gh) = \sum_{k=0}^n\left([x_0,\dots,x_k](g)\right)\left([x_k,\dots,x_n](h)\right)
    \end{equation*}
    \item \begin{equation*}
        [x_0,\dots,x_n]f = \int_0^1\mathrm{d}t_1\int_0^{t_1}\mathrm{d}t_2\cdots\int_0^{t_{n-1}}
        f^{(n)}\big(x_0+(x_1-x_0)t_1+\cdots+(x_n-x_{n-1})t_n\big)
        \mathrm{d}t_n
    \end{equation*}
\end{enumerate}


\paragraph{Peano核}设$\lambda(f):=\int_a^b f \mathrm{d}\mu$ 是$C[a,b]$上的连续线性泛函,并且与所有次数$\leq n-1$的多项式正交,即$\lambda(P_{n-1})=0$,则对任意$f\in C^n[a,b]$,有
\begin{equation*}
    \lambda(f) = \int_a^b f^{(n)}(t)\lambda\left[
        \frac{(\cdot-t)_+^{n-1}}{(n-1)!}
    \right]\mathrm{d}t
\end{equation*}

我们把由与多项式正交的泛函诱导出来的表达式
\begin{equation*}
    \lambda(f) = \int_a^b f^{(n)}(t)K(t)\mathrm{d}t
\end{equation*}
中的$K$称做Peano核.

当$x_0,\dots,x_n$互不相同时,差商运算作为一个泛函是连续的,因此可以诱导Peano核:

\begin{equation}\label{eq:差分算子的Peano核}
    [x_0,\dots,x_n]f = \int_a^b f^{(n)}(t)[x_0,\dots,x_n]\left[
        \frac{(\cdot-t)_+^{n-1}}{(n-1)!}\mathrm{d}t
    \right].
\end{equation}