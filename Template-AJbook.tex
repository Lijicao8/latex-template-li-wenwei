%!TEX TS-program = xelatex
%!TEX encoding = UTF-8

% Template for the class file AJbook, originally designed for Chinese book ``代数学方法''.
% Copyright 2018  李文威 (Wen-Wei Li).
% Permission is granted to copy, distribute and/or modify this
% document under the terms of the Creative Commons
% Attribution 4.0 International (CC BY 4.0)
% http://creativecommons.org/licenses/by/4.0/

% 使用自定义的文档类 AJbook.cls. 自动载入 xeCJK. 需要额外档案如下:
% font-setup-open.tex, coverpage.tex, titles-setup.tex, mycommand.sty, myarrows.sty
% 文档类选项 (key/val 格式):
% draftmark = true (未定稿, 底部显示日期) 或 false (成品), 默认 false,
% colors = true (链接带颜色无框) 或 false (黑色无框), 默认 true,
% traditional = true (繁体中文) 或 false (简体中文), 默认 false,
% coverpage = 封面档档名, 默认为空 (此时不制作封面), 一般是 .tex 档, 若为 *.pdf 的形式则直接引入 PDF 页面.
% fontsetup = 字体设置档档名,
% titlesetup = 章节格式设置档名.

% 注意: 如果文中未使用 \cite 和 \index 命令, 则可能报错.

% 需动用 imakeidx + xindy (两份索引), biblatex + biber (书目). 
% 索引用土法进行中文排序: 如 \index{zhongwen@中文} 等.
\documentclass[
	draftmark = true,   % 草稿模式下, 每页底部将打印相关版本信息.
%	traditional = true,
%	colors = false,
%	coverpage = coverpage.tex,
%	coverpage = coverpage.pdf,
%   geometry = b5,    % 版面设置, 目前仅容许 a4, b5, 默认 b5, 其它字串则不作自动设置.
	fontsetup = font-setup-open.tex,
	titlesetup = titles-setup.tex
]{AJbook}

% 可以修改章节编号的深度
% \setcounter{secnumdepth}{3}

% 必要时仅引入部分档案
% \includeonly{}

% 生成索引: 选用 xindy + imakeidx
\usepackage[xindy, splitindex]{imakeidx}
\makeindex[
	columns=2,
	program=truexindy,
	intoc=true,
	options=-M texindy -I xelatex -C utf8,
	title={名词索引}]	% 名词索引

\usepackage[unicode, bookmarksnumbered]{hyperref}	% 启动超链接和 PDF 文档信息所需
% 设置 PDF 文件信息
\hypersetup{
	pdfauthor = {李佶操},
	pdftitle = {AJbook 文档类模板},
	pdfkeywords = {Template},
	CJKbookmarks = true}

% 用 bibLaTeX 生成参考文献, 作为示例, 这里引入的是 Al-jabr.bib.
% 载入书目库: 文档类业已引入 biblatex + biber
\addbibresource{Al-jabr.bib}

% 增加表格高度
\renewcommand*\arraystretch{1.5}

% 自订公式的编号 (非必要)
\numberwithin{equation}{section}
\renewcommand{\theequation}{\thesection--\arabic{equation}}

% 自订 figure 的编号 (非必要)
%\numberwithin{figure}{section}
%\renewcommand{\thefigure}{\thechapter--\arabic{figure}}

% 自订 table 的编号 (非必要)
%\numberwithin{table}{section}
%\renewcommand{\thetable}{\thechapter--\arabic{table}}

\begin{document}
	\frontmatter	% 制作封面和目录.
	% 注意: 为了简化, 本模板不含封面. 请通过文档类的参数进行设置.
	
	\mainmatter		% 正文开始:逐章引入 TeX 代码
	% % % % % % % % % %
	\chapter{Properties of polynomials}
\section{Divided difference}
对于一个函数 $f$ 和点列
$X : x_{0}, \ldots, x_{n}$ (不要求按 $\leq$ 排序),我们定义 $f$ 的 $n$ 阶差商为
$$[x_{0}, \ldots, x_{n}]f := A_{n}$$
其中 $A_{n}$ 是插值多项式 $P_{n}(f, X; x)$ 的最高次项 $x^{n}$ 的系数。

\paragraph{n-阶差商满足的性质:}
\begin{enumerate}
    \item 若$f$高阶可微,则$[x_0,\dots,x_n]f$是$f^{(k)}(x_i),(0\leq k\leq m_i-1)$的线性组合,其中$m_i$是$x_i$的重数;
    \item $[x_0,\dots,x_n]f$是一个常数,若$f$是一个多项式,更进一步,若多项式的次数$<n$,则$n-$阶差商是0;
    \item$[x_0,\dots,x_0]f = f^{(n)}(x_0)/n!$.
    \item 牛顿型插值公式:\begin{equation*}
        P_n(f,X;x)= \sum_{k=0}^n(x-x_0)\dots(x-x_{k-1})[x_0,\dots,x_k]f;
    \end{equation*} 
    \item 若$f\in C^n[a,b]$,根据Rolle定理\begin{equation*}
        [x_0,\dots,x_n]f = f^{(n)}(\xi)/n!~~\text{for some } a\leq \xi\leq b;
    \end{equation*}
    \item 若$x_i$互异,则
    \begin{align}\label{eq:差分运算是线性组合}
        [x_0,\dots,x_n]f &= \sum_{k=0}^n\frac{f(x_k)}{(x_k-x_0)\cdots(x_k-x_{k-1})(x_k-x_{k+1})\cdots (x_k-x_n)}\notag\\
        & = \sum_{k=0}^n\frac{f(x_k)}{\Omega'(x_k)}
    \end{align}
    \item \begin{equation*}
        \Delta_h^n f(x) = n!h^n[x,x+h,\dots,x+nh]f;
    \end{equation*}
    \item \begin{equation*}
        [x_0,\dots,x_n](gh) = \sum_{k=0}^n\left([x_0,\dots,x_k](g)\right)\left([x_k,\dots,x_n](h)\right)
    \end{equation*}
    \item \begin{equation*}
        [x_0,\dots,x_n]f = \int_0^1\mathrm{d}t_1\int_0^{t_1}\mathrm{d}t_2\cdots\int_0^{t_{n-1}}
        f^{(n)}\big(x_0+(x_1-x_0)t_1+\cdots+(x_n-x_{n-1})t_n\big)
        \mathrm{d}t_n
    \end{equation*}
\end{enumerate}


\paragraph{Peano核}设$\lambda(f):=\int_a^b f \mathrm{d}\mu$ 是$C[a,b]$上的连续线性泛函,并且与所有次数$\leq n-1$的多项式正交,即$\lambda(P_{n-1})=0$,则对任意$f\in C^n[a,b]$,有
\begin{equation*}
    \lambda(f) = \int_a^b f^{(n)}(t)\lambda\left[
        \frac{(\cdot-t)_+^{n-1}}{(n-1)!}
    \right]\mathrm{d}t
\end{equation*}

我们把由与多项式正交的泛函诱导出来的表达式
\begin{equation*}
    \lambda(f) = \int_a^b f^{(n)}(t)K(t)\mathrm{d}t
\end{equation*}
中的$K$称做Peano核.

当$x_0,\dots,x_n$互不相同时,差商运算作为一个泛函是连续的,因此可以诱导Peano核:

\begin{equation}\label{eq:差分算子的Peano核}
    [x_0,\dots,x_n]f = \int_a^b f^{(n)}(t)[x_0,\dots,x_n]\left[
        \frac{(\cdot-t)_+^{n-1}}{(n-1)!}\mathrm{d}t
    \right].
\end{equation}
	\chapter{Splines}
\section{Definition and Simple properties}
\paragraph{样条定义}令$T^*:=(t_i^*)_1^s$或$T^*:=(t_i^*)_{=\infty}^\infty$是$\mathbb{R}$上严格递增的有限序列或双无穷序列,在第二种情况下,假定$|t_i^*|\to\infty,i\to\infty$.若在每个区间$(t_i^*,t_{i+1}^*)$上,函数$S$是一个次数$\leq m=r-1$的多项式,且至少有一个区间上次数恰为$m$,则称$S$是$r$阶且断点为$T^*$的样条.

\paragraph{样条在断点处光滑度}样条在断点$t_i^*$处的光滑度$m_i$定义如下:
\begin{itemize}
    \item 若$S$在$t_i^*$处不连续,则令$m_i=0$;
    \item 否则,$m_i$是满足$0<m_i\leq r$的最大整数,使得$S$在$t_i^*$的某个邻域内属于$C^{(m_i-1)}$,也就是说$S$的各阶导数直到$m_i-1$阶都连续.
\end{itemize}


\paragraph{样条空间}给定区间$A = [a,b]$或$A = \mathbb{R}$,以及断点序列$T^*$,可以在$A$上构造样条空间.记:$S_r^*(A)$
表示$A$上所有阶数$\leq r$的样条构成的空间;$S_r^*(T^*,A)$表示$A$上所有阶数$\leq r$且断点都包含在$T^*$中都样条构成的空间.

\paragraph{Schoenberg空间}
给定断点集$T^*$和一系列$m_i(0\leq m_i<r)$,我们定义$A$上的Schoenberg空间:他由所有阶数$\leq r$、断点包含在$T^*$中,并且$t_i^*$处的光滑度至少为$m_i$的样条$S$组成. 不过,通常不是直接使用$m_i$,而是使用所谓的亏格(defect)
\begin{equation*}
    k_i:= r-m_i,
\end{equation*}
因为$k_i$更好的描述了$S$在$t_i^*$处自由度的个数.这个Schoenberg空间记作
\begin{equation*}
    S_r:=S_r(R^*,\boldsymbol{k},A),\quad \boldsymbol{k}:=(k_i).
\end{equation*}

Schoenberg空间有如下性质
\begin{enumerate}
    \item $S_r(T^*,\boldsymbol{1},A)$是任意其他 Schoenberg空间$S_r(T^*,\boldsymbol{k},A)$的子空间;
    \item 若$S\in S_r(T^*,\boldsymbol{k},A)$,则其任意原函数$S_1\in S_{r+1}(T^*,\boldsymbol{k},A)$,其中亏格$\boldsymbol{k}$相同;
    \item $S_r(T^*,\boldsymbol{k},A)$总包含所有次数$\leq r-1$的多项式$P_{r-1}$,另一方面,他又是$S_r(T^*,\boldsymbol{r},A)$的子空间(即没有任何光滑性要求,只是分段多项式).
\end{enumerate}

\paragraph{样条空间的基}
在情形 $A=[a,b]$ 下,可以借助截断幂构造 $Y := S_r(T^*,\mathbf{k},A)$ 的一个自然基. 通常可以按以下方式找到一组基. 设$dim(Y)= N$,我们在 $Y$ 中找到元素
$S_1,\ldots,S_N$ 以及线性泛函 $a_1,\ldots,a_N$,满足
\begin{enumerate}
    \item \[
  a_j(S_i)=0,\quad i\neq j;\qquad a_j(S_j)=1,
\]
\item 
\[
  a_j(S)=0,\quad j=1,\ldots,N\Rightarrow S=0,
\]
\end{enumerate}
在这种情形下有
\begin{equation}\label{eq:dual-expansion}
  S = \sum_{j=1}^{N} a_j(S)\,S_j ,
\end{equation}
其中 $a_j$ 称为 $S_j$ 的\emph{对偶泛函}. 


\begin{theorem}[样条空间的基]\label{thm:样条空间的基}
若区间 $A=[a,b]$ 有限,则空间 $S_r(T^*,\mathbf{k},I)$ 具有如下基
\begin{align}
S_{-j}(x) &:= \frac{(x-a)^j}{j!}, && j = 0,\ldots,r-1, \label{eq:1.3a}\\
S_{i,j}(x) &:= \frac{(x-t_i^*)_+^{\,j}}{j!}, && j = r-k_i,\ldots,r-1,\quad i = 1,\ldots,s, \label{eq:1.3b}
\end{align}
其对应的对偶泛函为
\begin{align}
a_{-j}(S) &:= S^{(j)}(a), && j = 0,\ldots,r-1, \label{eq:1.4a}\\
a_{i,j}(S) &:= S^{(j)}(t_i^{*+}) - S^{(j)}(t_i^{*-}), && j = r-k_i,\ldots,r-1,\quad i = 1,\ldots,s. \label{eq:1.4b}
\end{align}
特别地,有
\begin{equation}
\dim S_r(T^*,\mathbf{k},A) = n + r,\qquad 
n := \sum_{i=1}^r k_i. \tag{1.5}
\end{equation}
\end{theorem}

\begin{proof}
令 $S\in S_r(T^*,\mathbf{k},I)$. 若对所有 $i=1,\ldots,S$ 以及 
$j = r-k_i,\ldots,r-1$ 都有 $a_{i,j}(S)=0$,则
\[
S,\ldots,S^{(r-1)}
\]
在每个 $t_i^*$ 处连续,因此 $S^{(r-1)}$ 为常数. 于是 $S^{(r)}\equiv 0$,
从而 $S$ 在 $A$ 上必为次数 $< r$ 的多项式. 若再对 $j=0,\ldots,r-1$ 都有
$a_{-j}(S)=0$,则该多项式所有阶导数在 $a$ 处均为零,只能是零函数,
即 $S\equiv 0$. 其余部分显然成立. 
\end{proof}


注:$k_i$表示每个节点处的自由度,对于多项式空间来说,基的个数谁$r$,样条空间每个节点$k_i$个自由度,也就是$k_i$个基.



根据定理\ref{thm:样条空间的基},$S\in S_r$,可以被写成
\begin{equation*}
    S(x) = P_{r-1}(x)+\sum_{i=1}^s\sum_{j=1}^{k_i}c_{i,j}(x-t_i^*)_+^{r-j}
\end{equation*}



\paragraph{样条基本不等式}
样条满足一些与多项式类似的基本不等式. 这里我们只给出 Nikol’skii 与 Markov 不等式在 Schoenberg 空间
$$
S_r(T^*,\mathbf{k},A),\qquad A=[a,b],
$$
中的变形,其中断点集合为
$$
T^* := \{t_j^*\}_{j=1}^s.
$$
我们约定
$$
t_0^* := a,\qquad t_{s+1}^* := b.
$$

\begin{theorem}
若断点 \(T^*\) 满足
\begin{equation}\label{eq:1.7}
\delta_0 \le \lvert t_{j+1}^* - t_j^* \rvert \le \delta,\qquad  j = 0,\dots,s,
\end{equation}
则对任意
$$
S\in S_r(T^*,\mathbf{k},A),\qquad A=[a,b],
$$
有
\begin{equation}\label{eq:1.8}
\lVert S\rVert_p \le C\,\delta^{1/q-1/p}\,\lVert S\rVert_q,\qquad
p_0 \le q \le p \le \infty,
\end{equation}
其中常数 \(C = C(p_0,r)\);并且当 \(k=1,\dots,r-1\) 时,还有
\begin{equation}\label{eq:1.9}
\lVert S^{(k)}\rVert_p \le C\,\delta_0^{-k}\,\lVert S\rVert_p,\qquad 0<p\le\infty,
\end{equation}
这里的常数 \(C = C(r)\). 
\end{theorem}

\section{B-样条}
定理\ref{thm:样条空间的基}中,把样条表示成若干截断幂级数之和,从一般观点看其实并不好用,因为截断幂级数的支撑集很大.因此需要引入另一组基,这就是B-样条(Basic splines).

\paragraph{B-样条定义}我们记$[x_0,\dots,x_n]f$表示$f$在这些点的$n$阶差商,我们定义B-样条为
\begin{equation*}
    M(x):=M(x;x_0,\dots,x_r):=r[x_0,\dots,x_r](\cdot -x)_+^{r-1}
\end{equation*}
其中的点$\{x_0,\dots,x_r\}$称为$M$的结点.

根据式\eqref{eq:差分算子的Peano核}可知,对任意$f\in W_1^r$,
\begin{equation*}
    [x_0,\dots,x_r]f = \frac{1}{r!}\int_\infty^\infty f^{(r)}(t)M(t)\mathrm{d}t.
\end{equation*}
令$f(t) = t^r$,可得
\begin{equation*}
    \int_{-\infty}^\infty M(t)\mathrm{d}t = 1.
\end{equation*}

当$r = 1$时,
\begin{equation*}
    M(x;x_0,x_1) = \frac{1}{x_1-x_0}\mathcal{X}_{(x_0,x_1)}(x),\quad x\neq x_0,x_1.
\end{equation*}


\paragraph{B-样条性质}
\begin{enumerate}
    \item $M(x)>0,\quad x\in(x_0,x_r);\quad M(x) = 0,\quad x\not\in [x_0,x_r]$
    \item $\begin{cases}
        M(x)\sim (x-x_0)^{r-k_0},~~x\to x_0^+;\\
        M(x)\sim (x-x_r)^{r-k_r},~~x\to x_r^-.
    \end{cases}\qquad\text{(其中$k_0,k_r$是$x_0,x_r$的重数.)}$
    \item $M(x)\leq C_r/(x_r-x_0),\quad x\in\mathbb{R},$
    \item 递推公式:\begin{equation*}
        M(x;x_0,\dots,x_r) = \frac{r}{r-1}\left[
            \frac{x-x_0}{x_r-x_0}M(x;x_0,\dots,x_{r-1})+\frac{x_r-x}{x_r-x_0}M(x;x_1,\dots,x_r)
        \right]
    \end{equation*}
\end{enumerate}
注:递推公式 4 可以由对$(\cdot - x)_+^{r-1}=(\cdot-x)(\cdot - x)_+^{r-2}$两侧使用差分的莱布尼茨公式得到.

我们得到一个$M$的变体,使得其具有更简单的递推公式,并且,其导数也是容易计算的:
\begin{equation}\label{eq:样条-N定义}
    N(x;x_0,\dots,x_r):=\frac{1}{r} (x_r-x_0) M(x;x_0,\dots,x_r) = (x_r-x_0)[x_0,\dots,x_r](\cdot-x)_+^{r-1}.
\end{equation}
则有
\begin{align*}
    N(x;x_0,\dots,x_r) &= \frac{x-x_0}{x_{r-1}-x_0}N(x;x_0,\dots,x_{r-1})+\frac{x_r-x}{x_{r}-x_1}N(x;x_1,\dots,x_{r}),\\[2pt]
    N'(x;x_0,\dots,x_r) & = -(r-1)\Big([x_1,\dots,x_r]-[x_0,\dots,x_{r-1}]\Big)(\cdot -x)_+^{r-2}\\[2pt]
     &= (r-1)\left(\frac{N(x;x_0,\dots,x_{r-1})}{x_{r-1}-x_0} - \frac{N(x;x_1,\dots,x_{r})}{x_{r}-x_1}\right)
\end{align*}
根据\eqref{eq:差分运算是线性组合},差分运算实际上是关于$f(x_k)$的线性组合.因此差分运算与求导运算可以交换顺序!除此之外$(t-x)_+^{r-1}$对于$x= t$点是$r-2$阶连续可微的.


\paragraph{整数节点B-样条}
特别重要的一种$B-$样条是整数节点的B-样条. 此时$M_1(x) = \mathcal{X}_I(x)$,其中$I = [0,1]$,并且$N_r = M_r: = M(x;0,\dots,r)$可以通过卷积得到:
\begin{equation*}
    M_r = M_{r-1}*\mathcal{X}_I =\mathcal{X}_I*\cdots *\mathcal{X}_I,
\end{equation*}
并且
\begin{equation*}
    M_r'(x) = N_r'(x) = N_r(x)-N_r(x-1).
\end{equation*}





\paragraph{样条函数随节点的变化}
\begin{lemma}
假设向量
\(
(y_0,\ldots,y_r)
\)
在 \(\mathbb{R}^{r+1}\) 中收敛到
\(
(x_0,\ldots,x_r)
\),并且 \(A\subset\mathbb{R}\) 是紧集.则有:

\begin{enumerate}
  \item[(i)] 若 \(x_0,\ldots,x_r\) 都是实数点,且其中属于集合 \(A\) 的点中,任意时刻至多有 \(r-1\) 个两两重合,则
  \[
    M(x; y_0,\ldots,y_r)\xrightarrow[y\to x]{} M(x; x_0,\ldots,x_r)
  \]
  在 \(x\in A\) 上一致收敛.函数 \(N(x; y_0,\ldots,y_r)\) 也有同样的结论.

  \item[(ii)] 若 \(0<q<\infty\),则
  \[
    N(x: y_0,\ldots,y_r)\xrightarrow[y\to x]{} N(x: x_0,\ldots,x_r)
  \]
  在空间 \(L_q(A)\) 的度量下收敛.
\end{enumerate}
\end{lemma}





\section{B-Spline Series}
Schoenberg空间$S_r(T^*,\boldsymbol{k},A)$在定理\ref{thm:样条空间的基}给出的基中相当简单,也能给出系数的简单公式. 然而,这种基有一些明显的缺点:计算样条$S$的系数时,需要知道$S$在所有断点$t_i^*$处的值. 对于大多数逼近或插值问题来说,一个局部的基更实用-B-样条基. 他只需要利用$S$在靠近$x$的$r+1$个点上的数据,就可以计算$S(x)$.

我们调整记号$S_r(T^*,\boldsymbol{k},A)$,其中$\boldsymbol{k}$表示断点$t_i^*$的重数,得到一个单调不减的序列
\begin{equation*}
    T = (t_i),
\end{equation*}
我们称$T$为Schoenberg空间的基本结点序列.,并把该空间记为
\begin{equation*}
    S_r:=S_r(T,A).
\end{equation*}
当$A = \mathbb{R}$时,若$|i|\to\infty$,则$|t_i|\to\infty$. 在所有情况下(因为$k_i\leq r$),都有
\begin{equation*}
    t_i<t_{i+r},\quad \forall i.
\end{equation*}
若$A = [a,b]$,我们还需要辅助结点
\begin{equation*}
    t_{-r+1\leq\cdots \leq t_0\leq a},\quad t_{n+r}\geq \cdots\geq t_{n+1}\geq b.
\end{equation*}
满足$k_i = 1$的结点为简单节点.

按第一节的定义,空间$S_r(T,A)$由所有阶数$\leq r$的样条$S$组成,这些样条在每个重数为$k_i$的基本结点处,都
具有阶数$<r-k_i$ 的连续导数. 
基本结点完全决定了空间$S_r(T,A)$;辅助结点只是在构造 B-样条基时需要. 


\paragraph{$S_r$的基}
\begin{theorem}
    $N_j(x):=N_{j,r}(x):=N(x;t_{j},\dots,t_{j+r}),~(j\in \Lambda)$是$S_r(T,A)$的一组基,其中
    \begin{equation*}
        \Lambda:=
        \begin{cases}
            \{-r+1,\dots,n\},&A = [a,b];\\
            \mathbb{Z},&A = \mathbb{R}.
        \end{cases}
    \end{equation*}
    $N_j(x)$定义见\eqref{eq:样条-N定义}.

    此外,对任意$S\in S_r(T,A)$,可以被唯一写成
    \begin{equation*}
        S(x) = \sum_{j\in\Lambda}c_j(S)N_j(x),
    \end{equation*}
    其中,% 定义 g_{j,r}
\[
g_{j,r}(x)
=
\begin{cases}
1, & r = 1,\\[4pt]
\dfrac{1}{(r-1)!}\displaystyle\prod_{\ell=1}^{r-1}\bigl(x - t_{j+\ell}\bigr), & r \ge 2,
\end{cases}
\]
并约定 \(g_j := g_{j,r}\).

% 定义 B 样条系数 c_j(S)
\[
c_j(S)
:= \sum_{\nu=0}^{r-1} (-1)^{\nu}\,
g_j^{(r-\nu-1)}(\xi_j)\, S^{(\nu)}(\xi_j),
\qquad S\in S_r(T,A),
\]
其中 \(\xi_j \in (t_j,t_{j+r})\cap A\).

\end{theorem}










\section{Good Approximation for $L^p$ by splines}


	% % % % % % % % % %
	\appendix
	\chapter{杂项 \texorpdfstring{$a+b$}{a+b}}
	按惯例, 附录以字母编号.
	
	\section{文字测试}
	龚自珍, \emph{乙亥杂诗}:
	\begin{enumerate}
		\item 其一
		\begin{center}
			掌故罗胸是国恩,小胥脱腕万言存。\\
			他年金匮如收采,来叩空山夜雨门。
		\end{center}
		\item 其二
		\begin{center}
			九州生气恃风雷,万马齐喑究可哀。\\
			我劝天公重抖擞,不拘一格降人才。
		\end{center}
		\item 其三
		\begin{center}
			吟罢江山气不灵,万千种话一灯青。\\
			忽然搁笔无言说,重礼天台七卷经。
		\end{center}
	\end{enumerate}	\index{gongzizhen@龚自珍}

	\begin{definition-theorem}[龚自珍]
		 《己卯京师作杂诗二首》:
		 \begin{center}
		 	文格渐卑庸福近,不知庸福究何如? \\
		 	常州庄四能怜我,劝我狂删乙丙书。
		 \end{center}
	\end{definition-theorem}

	交叉参照: 引理 \ref{prop:chen}.

	\section{测试: \texorpdfstring{$B_n(X)$}{BnX}}\label{sec:B}
	首先介绍 Bernoulli 多项式. 多项式变元记为 $X$.
	\begin{definition-proposition}\index{Bernoulli 多项式 (Bernoulli polynomials)}
		\emph{Bernoulli 多项式} $B_n(X) \in \mathbb{Q}[X]$ 由生成函数
		\begin{equation}
			\frac{t e^{tX}}{e^t - 1} = \sum_{n \geq 0} B_n(X) \cdot \frac{t^n}{n!} \; \in \mathbb{Q}[X][\![t]\!]
		\end{equation}
		确定. 称 $B_n := B_n(0)$ 为第 $n$ 个 \emph{Bernoulli 数}.
	\end{definition-proposition}

	\subsection{一张表格}
	以下来测试表格.

	\begin{table}[h!]
		\begin{equation*}\begin{array}{c|cccccccc}
			n & 0 & 1 & 2 & 4 & 6 & 8 & 10 & 12 \\ \hline
			B_n & 1 & -\frac{1}{2} & \frac{1}{6} & -\frac{1}{30} & \frac{1}{42} & -\frac{1}{30} & \frac{5}{66} & \frac{-691}{2730}
		\end{array}\end{equation*}
		\caption{前几个 Bernoulli 常数.}
		\label{table:B}
	\end{table}
	交叉参照: 练习 \ref{exo:Euler}.

	\begin{conjecture}[周恩来, 1917]\index{zhouenlai@周恩来}
		大江歌罢掉头东,邃密群科济世穷。面壁十年图破壁,难酬蹈海亦英雄。
	\end{conjecture}

	\begin{hypothesis}
		Riemann $\zeta$ 函数的非平凡零点全在 $\Re(s) = \frac{1}{2}$ 上.
	\end{hypothesis}

	引用测试: \cite{Oxl11, ZG}

	% % % % % % % % % %
	\backmatter
	% 使用 bibLaTeX 制作书目
	\printbibliography[heading=bibintoc]
	
	% 图, 表索引. 可有可无.
	%\listoffigures
	%\listoftables

	% 制作索引 (用 imakeidx 的功能可以制作多份)
	{\footnotesize
	\indexprologue{中文术语按汉语拼音排序.}
	\printindex}
\end{document}
